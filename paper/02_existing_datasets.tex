% Source of png : https://www.sciencedirect.com/topics/biochemistry-genetics-and-molecular-biology/rietveld-refinement

% Sources to mention
% Zenodo -> A good amount of raw powder data there but spread into smaller datasets; Inhomogenous, not specially prepared to be worked with by others, therefore difficult to work with especially at scale

% Crystal structures with corresponding simulated diffractograms;
% ->  ICSD (https://icsd.products.fiz-karlsruhe.de/) 
% - 230k structures;
% - Proprietary

% -> PDF (https://www.icdd.com/) -> 
%- 1200k "material data sets" (structures?) 
% + 19 k entries with "Experimental raw data digital patterns"; 
%  Proprietary
%The PDF itself lists the following data sources:

\begin{itemize}
\item Powder Diffraction File (ICDD)
\begin{itemize}
    \item ICDD Powders (00) – International Centre for Diffraction Data Newtown Square, PA, USA
    \item ICSD (01) – FIZ Karlsruhe – Leibniz Institute for Information Infrastructure (FIZ), Karlsruhe, Germany – Inorganic Crystal Structure Database
    \item NIST (03) – National Institute of Standards and Technology, Gaithersburg, Maryland, USA
    \item MPDS (04) – Material Phases Data System, Vitznau, Switzerland – Linus Pauling File (LPF)
    \item ICDD Single Crystal Data (05) – International Centre for Diffraction Data Newtown Square, PA, USA
\end{itemize}
\end{itemize}
 
% ->  COD (https://www.crystallography.net/cod/) -
% -  513k structures,
% - Fully open