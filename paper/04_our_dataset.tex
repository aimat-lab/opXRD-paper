% Notes:
% The subsections of this section should be the 
% individual contributions that we get from labs

% Sharing raw powder x-ray diffraction data: Challenges and benefits https://www.researchgate.net/publication/329256004_Sharing_powder_diffraction_raw_data_Challenges_and_benefits

In collaboration with several other institutions we have collected a dataset of diffractograms stemming from experiment, some of them labeled with corresponding structural information. \\
Currently, institutions that contributed towards our dataset include:
\begin{itemize}
    \item Institute of Nanotechnology at Karlsruhe institute of technology
    \item University of Southern California
    \item Lawrence Berkeley National Laboratory
\end{itemize}

We are still in the process of growing the dataset and additional contributions are still very welcome. 
To find out more about how to contribute to this dataset, visit our website specially designed for the purpose of collecting this dataset, https://xrd.aimat.science \\

A major advantage of this dataset over other comparable datasets is that is is very easy to handle in python and in PyTorch.
Our accompanying python library xrdpattern (https://github.com/aimat-lab/xrdpattern) provides a means to read in as much of the dataset as fits in memory simply by using PatternDB.load(data\_dirpath). The patterns attribute of this class is a list of the indivdual pattern diffractograms, each of which supports a standardize, plot and to tensor method. \\
The standardize method returns a "standardized" version of the pattern with a fixed angle range, a fixed number of entries and intensities normalized to the $[0,1]$ interval.