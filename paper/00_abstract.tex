



We, the AiMat group at the Karlsruhe Institute of Technology (KIT), are collecting a dataset of both labeled and unlabeled experimental powder X-ray diffraction (pXRD) data, which will be made publicly available once sufficient data has been collected. We look forward to publishing a paper surveying the database and discussing its uses, and would be happy to include anyone contributing data as co-authors. You can view the current draft here: State-of-the-database

The main motivation for collecting and publishing this dataset is to assist researchers working on machine learning based approaches for pXRD analysis. Any data, even unlabeled data, can provide clues how simulations can be made to better reflect real-world experiments and labeled data can be used to test the developed algorithms. We hope to provide both types of data in a wide variety and large quantity.

While we do not collect single crystal data, we are interested in any kind of polycrystalline samples, including thin films. If your samples exhibit preferred crystallite orientation, please make mention of this in the description field of the upload form. So far, we have received upwards of 4155 unique powder X-ray diffraction patterns from institutions including:



%Machine learning techniques have successfully been used to extract structural information such as the crystal space group from powder X-ray diffractograms.
%However, training directly on simulated diffractograms from databases such as
%the ICSD is challenging due to its limited size, class-inhomogeneity, and bias
%toward certain structure types. We propose an alternative approach of generating
%synthetic crystals with random coordinates by using the symmetry operations of
%each space group. Based on this approach, we 
%demonstrate online training of deep ResNet-like models on up to a few
%million unique on-the-fly generated synthetic diffractograms per hour. For our chosen task of
%space group classification, we achieved a test accuracy of 79.9\% on unseen ICSD
%structure types from most space groups. This surpasses the 56.1\% accuracy of the current state-of-the-art approach of training on ICSD crystals directly. Our results demonstrate that synthetically generated crystals can be
%used to extract structural information from ICSD powder diffractograms, which makes it possible to apply very large state-of-the-art machine learning models in the area of powder X-ray diffraction.
%We further show first steps toward applying our methodology to experimental data, where automated XRD data analysis is crucial, especially in high-throughput settings.
%While we focused on the prediction of the space group, our approach has the potential to be extended to related tasks in the future.