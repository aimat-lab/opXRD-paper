\documentclass[a4paper]{article}

\pdfoutput=1
\usepackage[utf8]{inputenc}
\usepackage[super,sort&compress,comma, square]{natbib}
%\renewcommand{\supercite}[1]{\cite{#1}}

%% Language and font encodings
\usepackage{tikz}
\usepackage{tablefootnote}
\usepackage{gensymb}
\usepackage[english]{babel}
\usepackage[T1]{fontenc}
\usepackage{csquotes}
\usepackage{doi}
%% Sets page size and margins
\usepackage[a4paper,top=3cm,bottom=2cm,left=3cm,right=3cm,marginparwidth=1.75cm]{geometry}

% Useful packages
\usepackage[version=4]{mhchem}
\usepackage{amsmath}
\usepackage{graphicx}
\usepackage{hyperref}
\hypersetup{colorlinks=true, allcolors=blue}
\usepackage{authblk}
\usepackage{multirow}
\usepackage{nicematrix,enumitem,booktabs}
\usepackage{nomencl}
\usepackage{booktabs}
\usepackage{tabularx}
\usepackage{textcomp}
\usepackage{amsfonts}
\usepackage{outlines} %%same
\usepackage{subfig}
\usepackage[version=4]{mhchem}
\usepackage{eurosym}
\usepackage[locale = US]{siunitx}
\usepackage{bm}
\usepackage[normalem]{ulem} %%temporary, for organization purposes
\usepackage{pythonhighlight}
\usepackage{graphicx}
\usepackage{subcaption}
\usepackage{adjustbox}

\usepackage{pifont}% http://ctan.org/pkg/pifont
\newcommand{\cmark}[1][]{\textcolor{green!80!black}{#1\quad\ding{52}}}
\newcommand{\xmark}[1][]{\textcolor{red}{#1\quad\ding{53}}}
\newcommand{\partialcheck}[1]{\textcolor{orange}{\quad\ding{52}(#1\%)}}
\captionsetup[table]{font=footnotesize}

\date{}
\newcommand{\red}[1]{\textcolor{red}{#1}}
\renewcommand\Authfont{\small\raggedright\itshape}  
\renewcommand\Affilfont{\normalfont\small}
\newcommand{\numpatterns}{92,552 }
\newcommand{\numlabeled}{2179 }

\DeclareSIUnit{\myeuro}{\text{\euro}}
\DeclareSIUnit\angstrom{\text {Å}}
\usepackage[final]{changes}

\interfootnotelinepenalty=10000
\title{opXRD: Open Experimental Powder X-ray Diffraction Database}

\newcommand{\ITI}{1}
\newcommand{\INT}{2}
\newcommand{\HKUSTGZ}{3}
\newcommand{\USC}{4}
\newcommand{\LBLMF}{5}
\newcommand{\LBLALS}{6}
\newcommand{\Hoffmann}{7}
\newcommand{\LBLCS}{8}
\newcommand{\EMPA}{9}
\newcommand{\KITICRT}{10}
\newcommand{\CPT}{11}
\newcommand{\KFUPMMSE}{12}
\newcommand{\KFUPMIRCIMR}{13}
\newcommand{\FZJIEMD}{14}

\author[\ITI,\INT]{Daniel Hollarek}
\author[\ITI,\INT]{Henrik Schopmans}
\author[\ITI,\INT]{Jona Östreicher}
\author[\ITI,\INT]{Jonas Teufel}
\author[\HKUSTGZ]{Bin Cao}
\author[\USC]{Adie Alwen}
\author[\INT]{Simon Schweidler}
\author[\LBLMF]{Mriganka Singh}
\author[\LBLMF,\LBLALS]{Tim Kodalle}
\author[\Hoffmann]{Hanlin Hu}
\author[\LBLCS]{Gregoire Heymans}
\author[\KFUPMMSE,\KFUPMIRCIMR]{Maged Abdelsamie}
\author[\EMPA]{Alexander Wieczorek}
\author[\EMPA]{Siarhei Zhuk}
\author[\CPT]{Arthur Hardiagon}
\author[\FZJIEMD]{Ruth Schwaiger}
\author[\CPT]{François-Xavier Coudert}
\author[\KITICRT]{Moritz Wolf}
\author[\EMPA]{Sebastian Siol}
\author[\LBLMF]{Carolin M. Sutter-Fella}
\author[\INT]{Ben Breitung}
\author[\USC]{Andrea M. Hodge}
\author[\HKUSTGZ]{Tong-yi Zhang}
\author[\ITI,\INT]{Pascal Friederich$^\ast$}

\affil[\ITI]{Institute of Theoretical Informatics, Karlsruhe Institute of Technology (KIT), 76131 Karlsruhe, Germany.  E-mail: pascal.friederich@kit.edu} % Informatik-Hauptgebäude, Am Fasanengarten 5, 76131 Karlsruhe
\affil[\INT]{Institute of Nanotechnology, Karlsruhe Institute of Technology (KIT), 76131 Karlsruhe, Germany} % Hermann-von-Helmholtz-Platz 1
\affil[\HKUSTGZ]{Guangzhou Municipal Key Laboratory of Materials Informatics, Advanced Materials Thrust, Hong Kong University of Science and Technology (Guangzhou) (HKUST), Guangzhou 511400, China} % 1 Du Xue Road,
\affil[\USC]{Department of Chemical Engineering and Materials Science, University of Southern California (USC), Los Angeles CA 90089, USA} % 925 Bloom Walk
\affil[\LBLMF]{Molecular Foundry Division, Lawrence Berkeley National Laboratory (LBNL), Berkeley 94720 CA, USA}  %67 Cyclotron Rd, Berkeley, CA 94720, United States
\affil[\LBLALS]{Advanced Light Source, Lawrence Berkeley National Laboratory,  Berkeley 94720 CA, USA}  % 6 Cyclotron Rd, Berkeley, CA 94720, United States
\affil[\Hoffmann]{Hoffmann Institute of Advanced Materials, Shenzhen Polytechnic, Shenzhen 518055, China} % Liuxian Street NO.7098, Shenzhen 518055, China
\affil[\LBLCS]{Lawrence Berkeley National Laboratory (LBNL), Chemical Sciences Division, Berkeley 94720 CA, USA}  % 1 Cyclotron Rd, Berkeley, CA 94720, United States
\affil[\EMPA]{Empa–Swiss Federal Laboratories for Materials Science and Technology (EMPA), 8600 Dübendorf, Switzerland}  % Ueberlandstrasse 129, 8600 Dübendorf, Switzerland
\affil[\KITICRT]{Engler-Bunte-Institut \& Institute of Catalysis Research and Technology, Karlsruhe Institute of Technology (KIT), Karlsruhe, Germany}  %Hermann-von-Helmholtz-Platz 1
\affil[\CPT]{Chimie ParisTech, PSL University, CNRS, Institut de Recherche de Chimie Paris, 75005 Paris, France} % 11 Rue Pierre et Marie Curie, 75005 Paris, France
\affil[\KFUPMMSE]{Material Science and Engineering Department, King Fahd University of Petroleum and Minerals (KFUPM), Dhahran 31261, Saudi Arabia} % 846R+8C2, Academic Belt Rd, KFUPM, Dhahran 34463, Saudi Arabia
\affil[\KFUPMIRCIMR]{Interdisciplinary Research Center for Intelligent Manufacturing and Robotics, King Fahd University of Petroleum and Minerals (KFUPM), Dhahran 31261, Saudi Arabia} %Academic Belt Road, Dhahran 31261, Saudi Arabia
\affil[\FZJIEMD]{Institute of Energy Materials and Devices, Forschungszentrum Juelich GmbH, 52425 Juelich, Germany} %Wilhelm-Johnen-Straße, 52428 Jülich

\begin{document}

\maketitle
\textbf{Keywords}: powder x-ray diffraction, machine learning, open-access data, data sharing, crystal structure determination, phase identification

\begin{abstract}
Powder X-ray diffraction (pXRD) experiments are a cornerstone for materials structure characterization.
Despite their widespread application, the analysis of pXRD diffractograms still presents a significant challenge to automation and thus a bottleneck in high-throughput experimentation automated materials discovery in self-driving labs.
Machine learning has emerged as a promising research direction to resolve this bottleneck by enabling automated powder diffraction data analysis.
A notable difficulty in applying machine learning to this domain is the lack of sufficiently sized experimental datasets, which has relegated machine learning researchers to train primarily on simulated data. Since simulations largely fail to accurately reflect the experiment, the performance of models trained on only simulated data often lacks transferability to experimental data and thus fails to provide value in practice.2
With the Open Experimental Powder X-Ray Diffraction Database (opXRD), we aim to remedy this by providing an openly available and easily accessible dataset of partially labeled experimental powder diffractograms, providing machine learning researchers with a large quantity of real experimental diffractograms collected from a broad range of samples.
We provide almost barrier-free software tools that allow experimental researchers to find their data and make it accessible - in virtually any widely used format.
We collected XXX labeled and YYY unlabeled diffractograms from a wide spectrum of materials classes, which establishes the first version of the opXRD database.
We hope that this ongoing effort can guide machine learning research toward fully automated analysis of pXRD data and thus enable future self-driving materials labs.
\end{abstract}

\newpage
\section{Introduction}\label{sec:Introduction}
% High throughput experimentation
The advent of high-throughput experiments holds the prospect of significantly accelerating the speed of materials discovery\cite{Liu2019}. The synthesis and characterization of novel materials are becoming increasingly efficient and automated, increasing the throughput of samples in experimentation pipelines\cite{MacLeod2019, Ludwig2019, Ozaki2020}.

% On Rietveld refinement
After fabricating a new material, a number of analysis techniques can be used to characterize the sample. One method that can be used for phase identification, phase quantification, grain size characterization, and to determine the crystal structure of a new material is powder X-ray diffraction (pXRD). 
When using pXRD measurements, crystal structures are typically determined through Rietveld refinement. In Rietveld refinement, an initial crystal structure model is fitted to the observed diffractogram by iteratively updating the structural model. Each update of the structural model seeks to minimize the difference between the observed diffractogram and the diffractogram simulated from the current structural model \cite{Dinnebier2019, Cano2021}. As Rietveld refinement is a local optimization method, the result of the refinement procedure is generally only as good as the initial structural model the process started from.

% Manual rietveld refinement and its problems
Manually performing Rietveld refinement is time-consuming and often requires expert knowledge. It is not scalable to the degree required to keep up with advances in throughput and efficiency in other steps of the experimentation pipeline. The refinement process requires the operator to determine an initial structural model from which the refinement can start and as well as initial values for parameters that characterize the background \cite{mccusker1999}. The structural model is usually obtained using search-match software, which identifies crystal structures with similar powder diffraction patterns from a database of crystal structures with accompanying powder diffraction patterns. However, an initial structural model obtained from such a database is not guaranteed to lead to an accurate structure solution through Rietveld refinement, especially not for novel structures. Additionally, attempting to refine all crystal structure parameters at once is known to lead to unphysical results\cite{Ozaki2020}. Hence parameters are refined iteratively, with each iteration only refining a limited set of parameters. Finding the correct order in which to refine structure parameters and finding the correct values for initial background parameters both present problems that add to the difficulty of the refinement process.

% Simulated pattern machine learning successes + Sim/Exp performance gap
Machine learning has the potential to speed up the manual analysis of powder diffractograms and keep pace with an automated high-throughput experimentation environment\cite{Agrawal2019, Surdu2023}.
Models can be either trained to predict crystal structure information directly given a diffractogram, or they can be used to automate the conventional refinement workflow. In the latter case a model would first predict an initial crystal structure \cite{Surdu2023} which is then refined by a second model trained to perform the refinement process \cite{Feng2019}. So far due to an absence of labeled datasets with experimental diffractograms\cite{Wang2020}, machine learning in this domain has largely relied on diffractograms simulated from known structures\cite{Park2017, Lee2023} or, most recently, from generated synthetic crystals\cite{Schopmans2023}. 
Models trained on datasets with simulated diffractograms have already shown strong performance in predicting phases \cite{Park2017,chenAutomatingCrystalstructurePhase2021, changProbabilisticPhaseLabeling2023}, lattice parameters\cite{Dong2021, Chitturi2021, Habershon2004, zhang2024crystallographic}, spacegroup \cite{cao2024simxrd, Schopmans2023, Oviedo2018, Park2017, Vecsei2018, Zaloga2020, Suzuki2020, Chakraborty2021,zhang2024crystallographic}, and crystallite size \cite{Dong2021, Chakraborty2021} from simulated diffractograms.
However, the performance substantially drops off when these models are applied to data originating from experiments \cite{cao2024simxrd, Schopmans2023,zhang2024crystallographic, Wang2020, Vecsei2018}. This discrepancy in performance arises due to imperfections in experimental data which are not present in diffraction patterns modeled under ideal conditions. This is discussed in more detail below.

Both labeled and unlabeled datasets of experimental powder diffractograms hold significant value for machine learning-based pXRD analysis, particularly with regard to bridging the performance gap between simulated and experimental domains. Labeled experimental data can be used to test and benchmark existing and new automated analysis approaches. This enables researchers to gauge how well a given model would perform under real-world conditions if integrated into an automated experimentation pipeline. Unlabeled experimental data enables machine learning researchers to evaluate how closely their simulations represent experimental data and modify their simulation algorithms accordingly. Unlabeled data can also find applications in transfer learning approaches to transfer model capabilities from the domain of simulated diffractograms to the domain of experimental diffractograms. While some experimental powder databases exist, their utility is limited by the fact that they are either small or not openly accessible.

In this work, we introduce an open powder X-ray diffraction (opXRD) database featuring a broad range of patterns collected from experiments. With a total of $92,552$ patterns collected from 6 contributing institutions, the opXRD database exceeds the size of the previously largest database of openly accessible experimental powder diffraction data by two orders of magnitude. To the best of our knowledge, the largest database of this type is the RRUFF database, containing 1290 experimental powder diffraction patterns \cite{lafuente2015}. Larger commerical datasets such as the PDF5+\cite{GatesRector2019} and the Linus Pauling File\cite{villars2018} exist, but their utility is limited by fees and restrictive licenses. License terms of commercial datasets, such as the PDF5+ and the Linus Pauling File, prohibit or restrict the publication of models trained on their data. In contrast, the opXRD database is both free and imposes no restrictions on how its data is used. Fig.~$\eqref{fig:ml_uses}$ provides an overview of machine learning workflows enabled and supported by the opXRD database.

% VISUAL ABSTRACT FIGURE
\begin{figure}[!htb]
    \centering
    \includegraphics[width=1.0\linewidth]{figures/pipeline.pdf}
    \caption{Experimental powder X-ray diffraction (pXRD) patterns from several contributors are collected in the \textit{opXRD} database. The proposed open-access database of experimental data aims to support each step in the pXRD-related machine learning workflow by informing better physics simulations, supplying model training data, and providing a foundation for realistic performance evaluations.}
    \label{fig:ml_uses}
\end{figure}

Of the \numpatterns patterns in the opXRD database \numlabeled patterns come with at least partial structural information of the underlying sample. Of these \numlabeled labeled patterns, more than 900 have full structural labels including atomic coordinates. This constitutes an experimental pXRD test dataset that is larger in size, richer in labels, and broader in represented experimental setups than the RRUFF database, which only provides lattice parameters as labels\cite{Armbruster2015}. However, since the majority of the opXRD database is unlabeled we also want to further discuss the uses of unlabeled data, including its role in improving pattern simulations and its application in transfer learning approaches.

The neglected effects that lead to discrepancies between simulated patterns and patterns stemming from experiments are largely known. Unaccounted effects may include preferred crystallite orientation, variations in grain size, crystal defects, the impact of temperature on the scattering process, internal stress, the non-monochromaticity of the X-ray source, and X-ray-induced fluorescence\cite{cao2024simxrd, Waseda2011, Pecharsky2023}. Additionally, varying experimental setups produce distinct powder diffraction patterns on the same sample. Features that may vary between experimental setups include the shape of diffraction peaks, the wavelength and polarization of the employed X-ray source, and the detector geometry \cite{cao2024simxrd, Waseda2011, Pecharsky2023}. The recorded scattering angles may also be slightly falsified if the sample is displaced from its intended position\cite{cao2024simxrd,hulbert2023}. As these and more neglected effects are integrated into the simulation process, real powder diffraction data can be used to evaluate how closely simulated data matches up with real data. While direct comparisons are only possible on labeled patterns, comparing the strength and prevalence of features between simulated and real data can nevertheless provide information about the fidelity of the simulation. Taking into account all neglected effects without making approximations will incur significant computational costs that will lower the size of the generated training data. A more efficient approach could be to use real experimental data to identify the effects that have the largest impact in practice and model them heuristically.

The second way in which unlabeled experimental data can serve to bridge the performance gap between simulated and experimental domains is through transfer learning. The objective of transfer learning is to transfer the capabilities of a model learned on a source domain in which labeled data is abundant to a target domain in which labeled data is sparse\cite{Zhuang2021}. In this context, the source domain is simulated powder diffraction patterns and the target domain is experimental powder diffraction patterns. Many approaches to transfer learning have been proposed, particularly in the domain of image classification \cite{Gatys2016, Ganin2015}.  These existing techniques can be adapted to facilitate transfer learning in the context of pXRD patterns. Seddiki \textit{et al}. have already successfully applied transfer learning in the domain of mass spectrometry to boost the accuracy of mass spectrum classification models\cite{Seddiki2020}. Since both mass spectrometry data and pXRD data are one-dimensional, this work demonstrates the merit of transfer learning in a setting similar to pXRD.

% Contributing + community driven
The opXRD database is intended as a growing, community-driven initiative. The database we present here is the first version, but we hope to further increase the database size through active engagement with the pXRD community. Our primary objective is to minimize the effort and thus the barrier to contributing experimental data to the opXRD database. Thus, we developed a program that helps to find and share data from pXRD lab computers. Users can select their most common pXRD file types, the program lists all files of that type, and users can select or deselect certain folders or files for sharing. Selected contributions will be uploaded to opXRD, processed to a common file format, and---if wanted---published on Zenodo on behalf of the contributors, before becoming part of the opXRD database. If labels are available, they can be shared with opXRD as well. Further details can be found on the opXRD website (\url{https://xrd.aimat.science/}). An overview of this process is given in Fig.~$\eqref{fig:submission}$ below.

\begin{figure*}[!htb]
    \centering
    \includegraphics[width=\linewidth]{figures/overview.pdf}
    \caption{Overview of the data collection pipeline. Datasets are submitted using an online submission form, optionally with the help of our submission helper software. After post-processing and data homogenization, we offer the creation of a Zenodo entry for each user submission and subsequently include the submission in the opXRD database.}
    \label{fig:submission}
\end{figure*}

As argued by Aranda and Kroon-Batenburg \textit{et al}.\cite{Aranda2018, Kroon-Batenburg2024}, sharing raw powder diffraction data is not only in the interest of furthering machine learning research but is also in line with open science principles. It furthers the ability of other researchers to reproduce published work and in turn, adds to the credibility of the publisher of the data. Compared to publishing data individually, publishing data on the opXRD database has the added benefit of contributing to a large, homogenous dataset with a standardized interface. This makes the data more easily accessible to other researchers and provides more value to researchers seeking large quantities of data. However, further data annotation with metadata is required to fully fulfill the FAIR data principles.

The opXRD database contains pXRD patterns from single and multiphase materials from a wide variety of material classes, including high-entropy materials, perovskites, and commercial catalysts. Some of the XRD data was collected on thin-films rather than on true powder samples, which may influence the quality of the data in regards to full structure resolution. Additionally, some of the data was collected in grazing-angle geometry rather than in the usual Bragg-Brentano geometry employed in powder diffraction.
The broad range of available experimental samples contained in the opXRD v1.0 database makes it possible to apply state-of-the-art ML approaches to the domain of pXRD analysis. We hope that the opXRD database can drive ML research in this field towards more advanced automated analysis workflows that can accelerate materials science research through ready application in high-throughput experimentation pipelines. Details of the experiments of research groups contributing to the opXRD database are discussed in Section~$\eqref{sec:our_dataset}$. A detailed description of how to acquire and use opXRD data is given in Section~$\eqref{sec:how_to_use}$, and Section~$\eqref{sec:summary_and_outlook}$ describes how further data can be contributed.
\subsubsection*{Review of machine learning-based pXRD analysis}
To showcase the need for datasets such as the one presented in this publication, we now discuss some recent approaches that apply machine learning methods to classification and regression tasks for powder diffractograms.

In 2020, Lee {\it et al.} trained a deep convolutional neural network (CNN) using simulated diffractograms based on structures from the ICSD, which is able to classify occurring phases in diffractograms of a specific compound pool \cite{Lee2020}. In 2022, they furthermore developed models based on fully convolutional neural networks and transformer encoders that predict the crystal system, the spacegroup, and other structural properties, such as the band gap \cite{Lee2022}. With their best model for the crystal system prediction on ICSD structures, they achieved a test accuracy of \SI{92.2}{\percent}. In 2017, Park {\it et al.} reached a test accuracy of roughly \SI{81}{\percent} for a CNN, which classifies space groups of simulated single-phase diffractograms \cite{Park2017}.

A regression analysis on lattice parameters within a broader framework encompassing all material classes was conducted by Chitturi {\it et al} \cite{Chitturi2021} in 2021. They developed a distinct CNN for each crystal system, utilizing a merged dataset from both the ICSD and the Cambridge Structural Database, and managed to achieve a mean absolute percentage error of about \SI{10}{\percent} for the lattice lengths, although they encountered difficulties in accurately predicting angles.
In 2024, Zhang {\it et al.} introduced a convolutional self-attention neural network trained on simulated patterns to classify crystal types \cite{zhang2024crystallographic}. Their model was tested on 23,073 unary, binary, and ternary inorganic crystal structures sourced from the COD. The study observed a noticeable performance drop when the pre-trained model was applied to real experimental patterns as opposed to simulated data. However, their recent work \cite{cao2024simxrd} proposes using convolutional peak descriptors that consider the detector's geometry, which reduces the performance gap in their benchmark tests.

Neural networks trained purely on experimental diffractograms can perform well when the range of samples is narrow and the data is collected only on a single machine \cite{Lee2023, hattrick-simpers2021}. However, in a more general setting with a wide range of investigated samples and employed diffractometers training neural networks purely on experimental diffractograms becomes infeasible. This is because of the limited availability of labeled experimental diffractograms relative to the scope of the task. However, in 2023, Salgado {\it et al.} \cite{Salgado2023} showed that adding a fraction of experimental patterns to a simulated training dataset improves the performance on unseen experimental patterns. They used \SI{50}{\percent} of the experimental patterns contained in the RRUFF database and added those to their large simulated training set. Then they tested their model's performance on the other half of the RRUFF database and achieved a performance increase in the 230-way spacegroup classification accuracy of \num{11} percentage points compared to the same model only trained on simulated patterns.

In 2024, Schuetzke {\it et al.} trained a classifier to classify if a diffractogram stems from an amorphous, single-phase, or multi-phase sample \cite{Schuetzke2024}. Due to the lack of experimental pXRDs, they built a pipeline to augment simulated diffractograms of a reference structure by, among other things, slightly varying the underlying crystal lattice. For spinel structures, they reported an accuracy of \SI{100}{\percent} but they also proved that their approach can be transferred to other datasets.

In 2023, Schopmans {\it et al.} presented an approach to generate synthetic crystal structures and their corresponding pXRD patterns on the fly during the training process \cite{Schopmans2023}. This approach defeats the issue of a limited dataset size, which limits the depth of neural networks that can be trained. However, the accuracy dropped substantially when we applied our space group classification model to experimental patterns from the RRUFF database. Augmenting our simulated patterns with background, noise, and impurities helps to bring simulated diffractograms closer to experimental ones, making models trained on them more performant on experimental diffractograms. However, this augmentation process could be improved by incorporating background and noise statistics from a broader experimental pXRD database, such as the one presented in this publication.

It becomes apparent that the more general the task is, the more challenging the transfer to experimental data becomes. For example, the space group classification task across all material systems is very general. Therefore, transferring it to the application on experimental diffraction patterns is difficult. \cite{Schopmans2023, Lee2022, Vecsei2018} On the other hand, there are some successful approaches that also work well on experimental data, but those are mostly methods that do phase determination in a limited compound space, making the task less complex \cite{Schuetzke2024, Lee2020}. 

The current volume of experimental pXRD patterns is insufficient to effectively train ML models, highlighting an urgent need for a comprehensive experimental pXRD database. The most advanced ML models currently are trained on approximately $10^5 - 10^6$ simulated diffractograms \cite{Salgado2023, Schopmans2023}. This is, to the best of our knowledge, two orders of magnitude larger than the largest currently curated experimental dataset, the PDF-5+ with approximately $2\cdot 10^4$ experimental patterns. It is even one order of magnitude larger than the approximately $10^5$ unlabeled diffractograms in the initial version of the opXRD dataset we present here.

To make ML-based pXRD data identification practical for experimental use and automate structure prediction despite lacking experimental training data two key approaches are essential. First, developing more sophisticated simulation methods to better approximate experimental patterns\cite{cao2024simxrd} by using statistics from experimental diffractograms. Second, creating an experimental database that enables transfer learning to bridge the gap between simulated and real-world data. For both of these steps, the development of opXRD is particularly significant, as it will provide a comprehensive experimental benchmark for the community, allowing fair comparison of baseline models and accurate evaluation of their applicability in real experimental situations.

\section{Existing datasets}

\subsubsection*{Existing experimental powder diffraction databases}
To contextualize opXRD within the current environment of experimental powder diffraction data, the following list provides an overview of the largest crystal structure databases that offer access to experimental powder diffraction data. For an overview of these databases refer to Tab.$\eqref{tab:exp_databases}$ below.\\

\begin{table}[!htb]
\centering
\caption{Overview of Experimental Powder Diffraction Databases}
\label{tab:exp_databases}
\scalebox{0.85}{
\begin{tabular}{@{}cccccccc@{}}
\toprule
\textbf{Name} & \textbf{No. of Patterns} & \textbf{O.A.} & \textbf{Comp.} & \textbf{Spg} & \textbf{Lattice} & \textbf{Unit Cell} & \textbf{Year est.} \\
\midrule
Linus Pauling file                     & 21,700                    & \xmark          & \cmark        & \cmark        & \cmark        & \cmark          & 2002            \\
Powder Diffraction File \tablefootnote{The PDF lists the Material Platform for Data Science (MPDS) as a database source. Since the MPDS is hosted by the Pauling File project, there is likely significant overlap in the experimental patterns available in the PDF and the Linus Pauling File.} & 20,800 & \xmark & \cmark & \cmark & \cmark & \cmark & 1941 \\
RRUFF                                  & 1290                      & \cmark          & \cmark        & \cmark        & \cmark        & \xmark          & 2006            \\
Crystallography Open Database          & 1063                      & \cmark          & \cmark        & \cmark        & \cmark        & \cmark          & 2003            \\
PowBase                                & 169                       & \cmark          & \xmark        & \xmark        & \xmark        & \xmark          & 1999            \\
\bottomrule
\end{tabular}}
\end{table}





\textbf{Linus Pauling File}:\cite{PaulingWeb} The Linus Pauling File is a largely commercial crystal structure database published and maintained by the Pauling File project \cite{villars2018}. It is currently distributed as Pearson Crystal data \cite{PearsonWeb} and the Materials Platform for Data Science (MPDS)\cite{MPDSWeb}. The database, first published in 2002, currently contains more than 534,000 crystal structures\cite{MPDSWeb} and 21,700 corresponding experimental powder diffraction patterns\cite{PearsonWeb}. 
This makes the Pauling file, to the best of our knowledge, the largest collection of experimental powder diffraction data available to researchers. As of November 2024, Pearson's crystal data is available to researchers through a purchase of a one-year license starting at a price point of \qty{2200}{\myeuro}. The MPDS is partially open, with the open MPDS data accessible through a web interface\cite{MPDSWeb}. Access to the full MPDS database can be purchased through a one-year license starting at \qty{2000}. \\

\textbf{Powder Diffraction File:} \cite{PDFWeb} The Powder Diffraction File (PDF), published and maintained by the International Center for Diffraction Data (ICDD), is a large collection of materials with accompanying powder diffraction data first published in 1941\cite{GatesRector2019}. According to the ICDD the PDF5+, the latest release of the PDF as of November 2024, contains over a million materials with accompanying powder diffraction data. However, most of these powder diffraction patterns are simulated. After inquiring with the ICDD in April 2024 we were told that only 20,800 of the powder diffraction patterns in the PDF5+ stemmed from experiments. Of these 20,800 entries, 10,954 contain information about the atomic coordinates of the underlying structures. Since the PDF5+ lists the MPDS as a database source, there is likely a significant overlap in the experimental patterns found in the PDF5+ and those found in the Pauling file. As of November 2024, the PDF5+ is available to researchers through a purchase of a one-year license starting at a price point of \$6265.00. However, the ICDD does not allow researchers to train machine learning models on PDF5+ data, regardless of whether the resulting models are published.\footnote{This information can be found in the product license agreement for the PDF5+ at \url{https://www.icdd.com/licensing-process/\#1528471154226-933e5cc6-8da7}.} \\

\textbf{RRUFF}: \cite{RRUFFWeb} The RRUFF Mineral Database, first published in 2006, provides detailed information on minerals, including their chemical compositions, crystallography, and spectroscopic data. \cite{lafuente2015} Managed by the University of Arizona, it was created to serve as a public repository for mineral identification and research. It contains \num{1290} powder diffraction patterns stemming from experiments each labeled with the lattice parameters and composition of the underlying structures. \\

\textbf{Crystallography Open Database:} \cite{CODWeb} The Crystallography Open Database (COD) is an open-access collection of crystal structures founded in 2003\cite{Graulis2009cod}. It currently provides over 500,000 crystal structures. Of these files, 1063 contain the experimental powder diffraction data that was used to determine the underlying crystal structures of the investigated samples. Hence, the experimental powder diffraction data contained in the COD is labeled with the full crystal structure information. \\

\textbf{PowBase}:\footnote{More information on PowBase can be found at \url{http://www.cristal.org/powbase/index.html}. PowBase was an initiative suggested in the Structure Determination by Powder Diffractometry (SDPD) mailing list which was hosted on the same site. The COD was another community initiative that grew out of this mailing list.} PowBase is a database of 169 mostly unlabeled experimental powder diffraction patterns collected and maintained by crystallography researcher Armel Le Bail starting in 1999. As of November 2024, all 169 patterns are still freely available for download. \\

There is also publicly available powder diffraction data uploaded to datasets on Zenodo. However, this data is split into disparate entries that typically only contain the work of a single research project. Additionally, extracting powder diffraction data at scale is hindered by the fact that the data is often given in plain text files in non-standardized formats, which are difficult to parse programmatically. We are currently planning a systematic large-scale extraction of powder diffraction data from databases like Zenodo with the help of a large language model. We will include this data in a future release of our dataset.\\

Aside from the databases mentioned above, we have also investigated several other crystal structure resources in search of experimental powder diffraction data. Crystal structure resources that were investigated but not found to contain any appreciable amount of publicly available experimental powder diffraction data include the Inorganic Crystal Structure Database \cite{ICSDWeb}, the Cambridge Structural Database \cite{CambridgeWeb}, the Materials Project database \cite{MatProjWeb}, the Crystallographic and Crystallochemical Database \cite{CrystallochemicalWeb}, the Bilbao Incommensurate Crystal Structure Database \cite{BilbaoWeb}, the Mineralogy Database \cite{MineralogyWeb}, the IUCr Raw data letters \cite{IUCrWeb}, the U.S. Naval Research Laboratory Crystal Lattice-Structures \cite{NRLWeb}, the Athena Mineral database \cite{AthenaWeb} and the Protein data bank\cite{PDBWeb}. This is to be expected as most structure solutions are achieved through single-crystal diffraction rather than powder diffraction. \\ 

%\pagebreak
\section{opXRD database}\label{sec:our_dataset}

% TODOS:
% - This paper talks about the importance of sharing raw powder X-ray diffraction data. I'd definitely cite it ->  \cite{Aranda2018}

%- The subsections of this section should be the 
% individual contributions that we get from labs

%------------------------------------------$

\begin{figure*}[!ht]
    \centering
    \missingfigure{} 
    \caption{Statistics, histograms, etc. of our dataset.}
    \label{fig:statistics}
\end{figure*}

In collaboration with several other institutions we have collected a dataset of diffractograms stemming from experiment, some of them labeled with corresponding structural information. 

\begin{itemize}
    \item Institute of Nanotechnology at Karlsruhe Institute of Technology
    \item University of Southern California
    \item Lawrence Berkeley National Laboratory
    \item Hong Kong University of Science and Technology (Guangzhou)
    \item Swiss Federal Laboratories for Materials Science and Technology
\end{itemize}

%% Paragraph by FX Coudert and Arthur Hardiagon
\subsection*{Chimie Paris Tech, PSL University}

We extracted experimental pXRD data from the Crystallography Open Database (COD).\cite{Grazulis2009, Vaitkus2023} The COD is, to our knowledge, the largest open-access collection of experimental crystal structures of organic, inorganic, and metal-organic compounds and minerals, containing more than 500,000 entries. \footnote{Available online at \url{https://www.crystallography.net/cod/}.} The data in the COD is placed in the public domain and licensed under the CC0 License. Of the entire COD database 5432 structures contained at least one tag from the {CIF\_POW} dictionary, i.e., a tag relating to powder diffraction studies. These 5432 structures only account for 1\% of the total COD database, but this is to be expected since most crystal structures are resolved from single-crystal diffraction. Of these 5432 files, most contained only metadata related to the powder diffraction experiment, but did not include the raw data of the pattern itself. We could extract raw, experimental pXRD patterns from 1063 files in total, after curation of a small number of files with clearly invalid data. \\

The pXRD data from the COD database are of high quality, with a median resolution of $\Delta(2\theta) = 0.013${\textdegree} and an average number of 9190 points measured per pattern. They span a wide chemical space, including organic, inorganic, and hybrid structures, including 75 different elements of the periodic table.

%% Paragraph by Ben Breitung
\subsection*{Institute of Nanotechnology, KIT}

The X-ray diffraction data were collected on a Bruker D8 Advance (Cu K$\alpha$1 radiation, $\lambda$ = 1.54056 Å). The samples were initially recorded for various research projects over the last ten years and were measured with different step sizes, times per step, and over different angle ranges, but all using Cu K$\alpha$1 radiation. The samples mostly contained transition metal oxides, sulfides, and fluorides. A major part of the research focused on high-entropy materials, which involved incorporating many different elements into single-phase structures, leading to peak shifts or phase separations. Most of those multi-component complex materials appeared in various structures, including rock-salt, spinel, fluorite, perovskite, and delafossite, to name a few.The samples were almost always prepared in powder form; therefore, powder XRD was performed on samples with adjusted height. To improve statistics, the samples were rotated during the entire measurement. Some air-sensitive samples were measured using a transparent polymer dome for protection. This dome led to increased background noise over the first 20° and slightly decreased pattern resolution. The samples were prepared using various synthesis techniques, mostly solid-state or wet chemical syntheses, to obtain the desired structures. Consequently, particle size and crystallinity varied significantly. The sample set also includes samples that were not successfully measured or where phases could not be identified. 

% Paragraph by Tim Kodalle
\subsection*{Lawrence Berkeley National Lab}

Experimental XRD data were collected at beamline 12.3.2 of the Advanced Light Source, the synchrotron at Lawrence Berkeley National Laboratory, using a photon energy of 10 keV (corresponding to a wavelength of $\lambda$ = 1.23984193 Å), selected using a Si(111) monochromator. Measurements were taken in grazing incidence geometry, i.e. using a beam incidence angle of 1$^{\circ}$. Two-dimensional diffraction images were recorded using a Dectris Pilatus 1M area detector at an angle between 34$^{\circ}$ and 36$^{\circ}$ with a sample-to-detector distance of roughly 190 mm. The two-dimensional data were calibrated using an Al$_{2}$O$_{3}$ calibration standard and integrated along the azimuthal angle. Data collection was performed in situ during thin-film deposition using a custom-made spin-coating and annealing stage. One dataset was collected spin-coating triple cation metal-halide perovskite precursor solutions with the composition Cs$_{0.05}$(MA$_{0.23}$FA$_{0.77}$)Pb$_{1.1}$(I$_{0.77}$Br$_{0.23}$)$_{3}$ (MA = Methylammonium, FA = Formamidinium) onto various substrates, including glass (amorphous), GaAs wafers (single crystalline) as well as stacks of glass/ITO, GaAs/Mo/Cu(In,Ga)Se$_{2}$/CdS/ZnO, and glass/Mo/Cu(In,Ga)Se$_{2}$/CdS/ZnO (glass/CIGS). Some of the substrates were additionally covered with a self-assembling monolayer of MeO-2PACz. The GaAs substrates were prepared by Dr. Jiro Nishinaga from the National Institute of Advanced Industrial Science and Technology (AIST) in Japan and the glass/CIGS substrates by Dr. Christian Kaufmann and his team at Helmholtz-Zentrum Berlin (HZB) in Germany. \\

A second dataset was collected from spin-coating metal-halide perovskite precursor solutions with varying compositions of \ce{MAPb(I_{1-x}Br_x)3} (where MA = Methylammonium and x = 0, 0.33, 0.5, 0.67, 1) spin-coated onto glass substrates. The substrates were preheated to different temperatures (30°C, 50°C, 70°C, and 90°C), and the spin-coating process was performed at a constant temperature on the preheated substrates. For both datasets, diffraction data were continuously measured during spin-coating, chemical induction of crystallization, and annealing of the samples (at 100$^{\circ}$C and 110$^{\circ}$C respectively) with a frequency of about 0.56 1/s and  0.54 1/s. Each in situ measurement consisted of about 500 to 1000 individual diffractograms. Depending on the substrate, each series of diffractograms shows an evolution from substrate only to a combination of polycrystalline perovskite, PbI$_{2}$ and substrate via several intermediate phases.


% Paragraph by Bin Cao
\subsection*{Guangzhou Municipal Key Laboratory of Materials Informatics, HKUST}

In the past two years, we established a small-scale experimental PXRD database called the X-Ray Phase Identification Public Experimental Dataset (XRed,  \href{https://github.com/WPEM/XRED}{https://github.com/WPEM/XRED}). The data in XRed is generated in our lab using instruments such as the Empyrean 3.0, Aeris, and Bruker D8 Advance diffractometers under copper targets. Hundreds of PXRD patterns have been refined, labeled with CIF files, and organized by elemental systems. The dataset includes original experimental files across single-phase to five-phase mixtures. This project is progressing toward integration with the comprehensive opXRD database to establish a large-scale, long-term experimental resource. 

% Paragraph by Moritz Wolf and Tony-yi 
\subsection*{Institute of Catalysis Research and Technology, KIT}

X-ray diffraction (XRD) was conducted with an X’Pert Pro MPD (Panalytical) in Bragg-Brentano geometry using a copper X-ray source ($\lambda_{K_{\alpha}} = 1.5406 Å$). The patterns were acquired from 5-80° $2\theta$ with a step size of 0.016711° or 0.033420° and a total acquisition time of 40 to 120 min. A variety of samples were analyzed including commercial catalysts, bulk reference materials, porous metal oxide particles, and nanoparticles. The latter were synthesized via the surfactant-free benzyl alcohol route (10.1039/C9DT01634A, 10.1016/j.matchemphys.2018.04.021). The cobalt oxide (CoO or Co3O4) and cerium oxide (CeO2) nanoparticles were in the size range of 4-16 nm according to the Scherrer equation. A series of porous Al2O3 materials, which were prepared by calcination of boehmite (AlOOH) at various temperatures, represents crystalline samples with limited long-range structure and various contributions of Al2O3 polymorphs.

% Paragraph by Alexander Wieczorek and Sebastiann Siol
\subsection*{Laboratory for Surface Science and Coating Technologies, Empa}

XRD data was measured using a Bruker D8 Discover equipped with a Cu~K$\alpha$ radiation source ($\lambda$~=~1.54056~{\AA} for Cu~K$\alpha$\textsubscript{1} and $\lambda$~=~1.54439~{\AA} for Cu~K$\alpha$\textsubscript{2}) in a Bragg-Brentano geometry. For the reported data sets the instrument was equipped with a Goebel mirror effectively removing the Cu~K$\beta$ radiation. The data set originates from the combinatorial exploration of the Zn–V–N compositional space, as well as data gathered from multiple research activities on more established metal halide perovskite semiconductors. All data was collected from thin films deposited on borosilicate glass. The Zn–V–N films showed some preferential out of plane orientation, while for the perovskites the preferential orientation was minimal, resulting in the presence of all reflections.
Each combinatorial Zn–V–N library was synthesized using radio-frequency co-sputtering of Zn and V in a mixed Ar and N\textsubscript{2} plasma. An orthogonal deposition temperature and composition gradient was created, resulting in a deposition temperature of 220~{\textdegree}C for samples 1~–~9 and 114~{\textdegree}C for samples 37~–~45. The composition for each sample was determined using X-Ray fluorescence (XRF) spectroscopy which was further calibrated through Rutherford backscattering spectroscopy (RBS) based on selected samples. The newly identified and isolated semiconductor Zn\textsubscript{2}VN\textsubscript{3} was identified to exhibit a cation-disordered wurtzite structure as verified by additional GI-XRD and SAED measurements.\cite{Zhuk2021}
Tin halide perovskites were deposited using single-step spin-coating as reported elsewhere.\cite{Wieczorek2023} Methylammonium lead iodide libraries with varying degrees of residual PbI\textsubscript{2} were deposited using a two-step procedure involving both thermal evaporation of PbI\textsubscript{2} and subsequent spin-coating of a methylammonium solution. The relative phase fractions were quantified using supplementary azimuthal angle scans coupled with structural factors and geometrical factors as reported elsewhere.\cite{Wieczorek2024} Fully inorganic lead perovskite libraries were prepared using thermal co-evaporation of lead and cesium halide salts.
All metal halide perovskite libraries were measured within a custom-made X-Ray transparent inert-gas dome, resulting in the presence of minor additional features within the $\theta$~=~19~–~31{\textdegree} range. For all combinatorial libraries where any phases are specified, the complete set of phases are reported in the meta data.

We are still in the process of growing the dataset and additional contributions are still very welcome. To find out more about how to contribute to this dataset, visit our website specially designed to collect this dataset, \url{https://xrd.aimat.science} .

\section{Usage}\label{sec:how_to_use}
\subsubsection*{How to contribute and how to use}

TODO: A small tutorial with a few python commands to show how easy it is to use our dataset through the provided API.

A major advantage of this dataset over other comparable datasets is that it is very easy to handle in python and in PyTorch.
Our accompanying python library xrdpattern (\url{https://github.com/aimat-lab/xrdpattern}) provides a means to read the dataset simply by using PatternDB.load(data\_dirpath). The patterns attribute of this class is a list of the indivdual pattern diffractograms, each of which supports a standardize, plot and to tensor method. \\
The standardize method returns a "standardized" version of the pattern with a fixed angle range, a fixed number of entries and intensities normalized to the $[0,1]$ interval.

% 
We have launched Xqueryer (\url{http://xqueryer.caobin.asia/}), a cutting-edge, free online platform for intelligent structure identification. Users can upload experimental pXRD data to identify crystal structures and extract all relevant information available in the Materials Project database. All usage complies with Xqueryer's terms of service. Uploaded data are securely stored and integrated into the opXRD database.

\section{Summary and Outlook}\label{sec:summary_and_outlook}
With the opXRD dataset, a curation of XXX unlabeled and YYY labeled experimental powder XRD patterns from a wide range of different materials systems, we provide the largest currently available source of experimental XRD patterns. With this, we address the need for experimental data that arises when developing algorithms and analysis tools for XRD data, both based on machine learning and classical approaches. The data can be used for the actual method development and for testing. Our dataset is a valuable and so far missing resource to drive further developments in the automated analysis of XRD data.

% How to contribute
Rather than a finished project, the opXRD dataset is an ongoing effort to collect experimental powder XRD data. We invite everyone in possession of experimental powder XRD data to submit it to our dataset, to further improve the dataset and aid further developments in this field. Our submission page (\url{https://xrd.aimat.science/}) and submission helper software will be kept available to collect more data. We will keep updating and maintaining the dataset with incoming new submissions.


\subsection*{Supporting Information}
Supporting Information is available from the Wiley Online Library or the author.

\subsection*{Data availability}
The opXRD database is available on Zenodo at \url{https://zenodo.org/records/14254270}. It is published under the Creative Commons Attribution 4.0 International license.%\footnote{Detailed terms of this license can be found at \url{https://creativecommons.org/licenses/by/4.0/deed.en}.}
It can be downloaded by any user without any barriers or restrictions. For further details, please refer to Section~$\eqref{sec:how_to_use}$.

\subsection*{Conflicts of interest}
There are no conflicts of interest to declare.

\subsection*{Acknowledgements}
We acknowledge financial support by the German Research Foundation (DFG) through the Research Training Group 2450 “Tailored Scale-Bridging Approaches to Computational Nanoscience”. We acknowledge support by the Federal Ministry of Education and Research (BMBF) under Grant No. 01DM21001B (German-Canadian Materials Acceleration Center). We acknowledge financial support from the Helmholtz Foundation Model Initiative within Project "SOL-AI". Part of this work was funded under the France 2030 framework by Agence Nationale de la Recherche (project ANR-22-PEXD-0009 of PEPR DIADEM). Work at the Molecular Foundry was supported by the Office of Science, Office of Basic Energy Sciences, of the U.S. Department of Energy under Contract No. DE-AC02-05CH11231. Work at the Advanced Light Source (ALS) was done at beamline 12.3.2. The ALS is a DOE Office of Science User Facility under contract no. DE-AC02-05CH11231. The development of the online phase identification platform is supported by the Guangzhou-HKUST(GZ) Joint Funding Program (No. 2023A03J0003). Work by the USC group was supported by the National Science Foundation (NSF) grant numbers DMR-2227178 and OISE-2106597. Funding by the Helmholtz Research Program “Materials and Technologies for the Energy Transition (MTET), Topic 3: Chemical Energy Carriers" is kindly acknowledged. Work by the Empa group was supported by the Strategic Focus Area–Advanced Manufacturing (SFA–AM) through the project Advancing manufacturability of hybrid organic–inorganic semiconductors for large area optoelectronics (AMYS) as well as the Empa internal research call 2020.

%\subsection*{Table of contents}
%\tableofcontents

%A short text and graphic should be provided for the Table of Contents (ToC). The ToC text should describe the main results in 50 to 60 words. It should be written for a general audience and be written in the third person.

%The ToC figure should convey the main message of the article. It does not have to be a figure from the article; it can be a combination of figures or a new, original figure composed to represent the topic. The author must be the copyright holder for this figure and any images used to create it. The size of the image should be either 55 mm × 50 mm (w × h) or 110 mm × 20 mm (w × h). 

\printnomenclature

\clearpage

\bibliographystyle{bibstyle/bibstyle.bst}
\bibliography{references}

\setcounter{section}{0}
\renewcommand{\thesection}{S\arabic{section}}
\setcounter{figure}{0}
\renewcommand{\thefigure}{S\arabic{figure}}
\setcounter{table}{0}
\renewcommand{\thetable}{S\arabic{table}}

\section*{Supporting Information}
%Supporting Information is available from the Wiley Online Library or the author.

\section*{Description of opXRD files on Zenodo}
The database comes in two zip archives, ``opxrd.zip'' and ``opxrd\_in\_situ.zip''. The latter contains the in-situ data with highly correlated patterns recorded through time series measurements. Within the .zip archives patterns are saved as .json files grouped in folders indicating the contributing institution. If an institution contributed data from several projects, the contributed data is further divided into folders indicating the research project. These research project folders are labeled alphabetically in the order they are introduced in section $\eqref{sec:our_dataset}$. 
Each .json file contains a pattern recorded from an X-ray diffraction experiment. If available, the composition and structure of the investigated sample and experiment conditions are also included in this file. Patterns belonging to time series measurements are labeled with filenames that indicate the measurement series they belong to as well as their order in that series. 


\section*{opXRD Python library usage}

The opXRD Python library allows the dataset to be accessed through one simple command: \pyth{OpXRD.load(root_dirpath)}. If the database is locally available under \pyth{root_dirpath} this command loads the library from this location. If the database is not available locally at this location, the database is automatically downloaded to \pyth{root_dirpath}. 


\section*{Combined pattern plots}
The Figure below $\eqref{fig:combined}$ shows 50 randomly selected samples of the X-ray diffraction patterns found in each of the research projects contributed to the opXRD database.

\begin{figure}[!htb]
    \centering
    \includegraphics[width=\linewidth]{figures/combined.png}
    \caption{50 randomly chosen X-ray diffraction patterns from each contributed dataset. The figure shows data from the following datasets: a) EMPA, b) LBNL-A, c) LBNL-B, d) LBNL-C, e) USC, f) INT, g) HKUST-A, h) HKUST-B, i) CNRS, j) IKFT.}
    \label{fig:combined}
\end{figure}

\pagebreak


\end{document}
