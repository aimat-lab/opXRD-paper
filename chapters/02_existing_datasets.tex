\pagebreak

\subsubsection*{Existing experimental powder diffraction databases}
To contextualize opXRD within the current environment of experimental powder diffraction data, the following list provides an overview of the largest crystal structure databases that offer access to experimental powder diffraction data. For an overview of these databases refer to Tab.$\eqref{tab:exp_databases}$ below.\\

\begin{table}[!htb]
\centering
\caption{Overview of Experimental Powder Diffraction Databases}
\label{tab:exp_databases}
\begin{tabular}{@{}lcc@{}}
\toprule
\textbf{Name}                          & \textbf{No. of Patterns} & \textbf{Year Established} \\
\midrule
Powder Diffraction File (PDF)          & 19,000                  & 1941                      \\
Pearson's Crystal Data                  & 21,700                  & 2007        \\
Crystallography Open Database (COD)    & 1063                    & 2003                      \\
RRUFF                                  & 1290                    & 2006                      \\
PowBase                                & 169                     & 1999                      \\
\bottomrule
\end{tabular}
\end{table}

\textbf{Powder Diffraction File:} \cite{PDFWeb} The Powder Diffraction File (PDF), published and maintained by the International Center for Diffraction Data (ICDD), is a large collection of materials with accompanying powder diffraction data first published in 1941\cite{GatesRector2019}. According to the ICDD the PDF5+, the latest release of the PDF as of November 2024, contains over a million materials with accompanying powder diffraction data. However, most of these powder diffraction patterns are simulated. After inquiring with the ICDD in April 2024 we were told that only 19,000 of the powder diffraction patterns in the PDF5+ stemmed from experiments. Of these 19,000 entries, 10,954 contain information about the atomic coordinates of the underlying structures.\footnote{We reached out to the contact address given on their website, info@icdd.com.} Still, this makes the PDF5+, to the best of our knowledge, the largest collection of experimental powder diffraction data available to researchers. As of November 2024, the PDF5+ is available to researchers through a purchase of a one-year license starting at a price point of \$6265.00.\\


\textbf{Pearson's Crystal data}:\cite{PearsonWeb} Pearson's crystal data is a commercial database of 395,000 crystal structures including 21,700 experimental powder diffraction patterns maintained by ASM International first published in 2007\cite{PaulingWeb}.  As of November 2024, Pearson's crystal data is available to researchers through a purchase of a one-year license starting at a price point of 2200 \euro.\\

% TODO: I need to check that by asking
%Pearson's crystal data is provided by the Pauling File project which also provides the Materials Phases Data System database (MPDS). 
%Since the MPDS is listed as a database source for the PDF5+ \cite{kaduk2007}, there is likely a significant overlap between the experimental powder diffraction data found in Pearson's crystal data and that found in the PDF. 
 %\tablefootnote{The PDF and Pearson's Crystal data both derive some of their data from the Linus Pauling File, which has a similar number of experimental powder diffraction patterns as both the PDF and Pearson's Crystal Data. Hence there is likely significant overlap in the patterns available in the PDF and in Pearson's crystal data.}

\textbf{Crystallography Open Database:} \cite{CODWeb} The Crystallography Open Database (COD) is an open-access collection of crystal structures founded in 2003\cite{Graulis2009cod}. It currently provides over 500,000 crystal structures in the form of .cif files. Of these files, 1063 contain the experimental powder diffraction data that was used to determine the underlying crystal structures of the investigated samples. Hence the experimental powder diffraction data contained in the COD is fully labeled. The .cif files include both lattice parameters and the contents of the unit cell of the underlying structure. \\

\textbf{RRUFF}: \cite{RRUFFWeb} The RRUFF Mineral Database, first published in 2006, provides detailed information on minerals, including their chemical compositions, crystallography, and spectroscopic data. \cite{lafuente2015} Managed by the University of Arizona, it was created to serve as a public repository for mineral identification and research. It contains \num{1290} powder diffraction patterns stemming from experiments each labeled with the lattice parameters and composition of the underlying structures. \\

\textbf{PowBase}:\footnote{More information on PowBase can be found under \url{http://www.cristal.org/powbase/index.html}. PowBase was an initiative suggested in the Structure Determination by Powder Diffractometry (SDPD) mailing list which was hosted on the same site. The COD was another community initiative that grew out of this mailing list.} PowBase is a database of 169 mostly unlabeled experimental powder diffraction patterns collected and maintained by crystallography researcher Armel Le Bail starting in 1999. As of November 2024, all 169 patterns are still freely available for download. \\


There is also publicly available powder diffraction data uploaded to datasets on Zenodo. However, this data is split into disparate entries that typically only contain the work of a single research project. Additionally, extracting powder diffraction data at scale is hindered by the fact that the data is often given in plain text files in non-standardized formats, which are difficult to parse programmatically. \\

Aside from the databases mentioned above, we have also investigated several other crystal structure resources in search of experimental powder diffraction data. Crystal structure resources that were investigated but not found to contain any appreciable amount of publicly available experimental powder diffraction data include the Inorganic Crystal Structure Database,   Cambridge Structural Database,  the Materials Project database,  the Crystallographic and Crystallochemical Database, the Bilbao Incommensurate Crystal Structure Database,  the Mineralogy Database, IUCr Raw data letters, the U.S. Naval Research Laboratory Crystal Lattice-Structures, the Athena Mineral database and the Protein data bank.  This is to be expected as most structure solutions are achieved through single-crystal diffraction rather than powder diffraction.\\ 

