TODO: A small tutorial with a few python commands to show how easy it is to use our dataset through the provided API.

A major advantage of this dataset over other comparable datasets is that it is very easy to handle in python and in PyTorch.
Our accompanying python library xrdpattern (\url{https://github.com/aimat-lab/xrdpattern}) provides a means to read the dataset simply by using PatternDB.load(data\_dirpath). The patterns attribute of this class is a list of the indivdual pattern diffractograms, each of which supports a standardize, plot and to tensor method. \\
The standardize method returns a "standardized" version of the pattern with a fixed angle range, a fixed number of entries and intensities normalized to the $[0,1]$ interval.

% caobin
We have launched Xqueryer (http://xqueryer.caobin.asia/), a free online platform for phase identification. Users can upload experimental PXRD data to identify and match structures from the Materials Project database. Uploaded data is stored and integrated into our opXRD database.