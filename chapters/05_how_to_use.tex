The opXRD database is hosted on Zenodo\footnote{The dataset can be found at \url{https://zenodo.org/records/14278656}.} and can be downloaded by any user without any barriers or restrictions. The database comes in two zip archives, ``opxrd.zip'' and ``opxrd\_in\_situ.zip''. The latter contains the in-situ data which contains highly correlated patterns recorded through time series measurements. Within the .zip archives patterns are saved as .json files grouped in folders indicating the contributing institution. If an institution contributed data from several projects, the contributed data is further divided into folders indicating the research project. These research project folders are labeled alphabetically in the order they are introduced in section $\eqref{sec:our_dataset}$. Each .json file contains a pattern recorded from an X-ray diffraction experiment. If available, the composition and structure of the investigated sample and experiment conditions are also included in this file. Patterns belonging to time series measurements are labeled with filenames that indicate the measurement series they belong to as well as their order in that series. \\

Next to the availability of the opXRD dataset on Zenodo, we also provide a Python library ``opxrd'' to easily download and interface with the dataset. The instructions for how to install this library can be found in the repository associated with the library.\footnote{The repository to this library can is located at \url{https://github.com/aimat-lab/opxrd}.} The opxrd library allows the dataset to be accessed through one simple command: \pyth{OpXRD.load(root_dirpath)}. If the database is locally available under \pyth{root_dirpath} this command loads the library from this location. If the database is not available locally at this location, the database is automatically downloaded to \pyth{root_dirpath}. \\

Our library further includes options for standardization, plotting, and the conversion to \emph{PyTorch} tensors. We provide a Jupyter Notebook\footnote{This notebook can be found here: \url{https://colab.research.google.com/github/aimat-lab/opXRD/blob/main/opxrd/usage.ipynb}.} that showcases these functionalities in more detail. This notebook also illustrates how to interface with the OpXRD database through Python. 