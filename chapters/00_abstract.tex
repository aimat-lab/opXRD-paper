Powder X-ray diffraction (pXRD) experiments are a cornerstone for materials structure characterization.
Despite their widespread application, the analysis of pXRD diffractograms still presents a significant challenge to automation and thus a bottleneck in high-throughput experimentation automated materials discovery in self-driving labs.
Machine learning has emerged as a promising research direction to resolve this bottleneck by enabling automated powder diffraction data analysis.
A notable difficulty in applying machine learning to this domain is the lack of sufficiently sized experimental datasets, which has relegated machine learning researchers to train primarily on simulated data. Since simulations largely fail to accurately reflect the experiment, the performance of models trained on only simulated data often lacks transferability to experimental data and thus fails to provide value in practice.2
With the Open Experimental Powder X-Ray Diffraction Database (opXRD), we aim to remedy this by providing an openly available and easily accessible dataset of partially labeled experimental powder diffractograms, providing machine learning researchers with a large quantity of real experimental diffractograms collected from a broad range of samples.
We provide almost barrier-free software tools that allow experimental researchers to find their data and make it accessible - in virtually any widely used format.
We collected XXX labeled and YYY unlabeled diffractograms from a wide spectrum of materials classes, which establishes the first version of the opXRD database.
We hope that this ongoing effort can guide machine learning research toward fully automated analysis of pXRD data and thus enable future self-driving materials labs.